%%%%%%%%%%%%%%%%%%%%%%%%%%%%%%%%%%%%%%%%%%%%%%%%%%%%%%%%%%%%%%%%%%%%%%%%%%%%%%%%
%% Origin of this template:
%% https://authors.acm.org/proceedings/production-information

%% Commands for TeXCount
%TC:macro \cite [option:text,text]
%TC:macro \citep [option:text,text]
%TC:macro \citet [option:text,text]
%TC:envir table 0 1
%TC:envir table* 0 1
%TC:envir tabular [ignore] word
%TC:envir displaymath 0 word
%TC:envir math 0 word
%TC:envir comment 0 0

%% DOCUMENT CLASS %%%%%%%%%%%%%%%%%%%%%%%%%%%%%%%%%%%%%%%%%%%%%%%%%%%%%%%%%%%%%%
\documentclass[sigconf]{acmart}
% \documentclass[manuscript, screen, review]{acmart}


%% PACKAGES %%%%%%%%%%%%%%%%%%%%%%%%%%%%%%%%%%%%%%%%%%%%%%%%%%%%%%%%%%%%%%%%%%%%
\usepackage{csquotes}


%% SETTINGS %%%%%%%%%%%%%%%%%%%%%%%%%%%%%%%%%%%%%%%%%%%%%%%%%%%%%%%%%%%%%%%%%%%%

% Remove identation on "double-linebreaks".
\setlength{\parindent}{0pt}


%% LICENSE %%%%%%%%%%%%%%%%%%%%%%%%%%%%%%%%%%%%%%%%%%%%%%%%%%%%%%%%%%%%%%%%%%%%%

%% Remove all the licence, conference, ISBN and DOI information that are the
%% default for this template.
\settopmatter{printacmref=false} % Removes citation information below abstract
\settopmatter{printfolios=true} % Adds page numbers (optional, for review)
\renewcommand\footnotetextcopyrightpermission[1]{} % Removes conference footnote


%% CITATION %%%%%%%%%%%%%%%%%%%%%%%%%%%%%%%%%%%%%%%%%%%%%%%%%%%%%%%%%%%%%%%%%%%%

%% The majority of ACM publications use numbered citations and references.
%% The command \citestyle{authoryear} switches to the "author year" style.
%%
%% If you are preparing content for an event sponsored by ACM SIGGRAPH, you must
%% use the "author year" style of citations and references.
%% Uncommenting the next command will enable that style.
% \citestyle{acmauthoryear}

%% BEGIN %%%%%%%%%%%%%%%%%%%%%%%%%%%%%%%%%%%%%%%%%%%%%%%%%%%%%%%%%%%%%%%%%%%%%%%
%% End of the preamble, start of the body of the document source.
\begin{document}

%% TITLE %%%%%%%%%%%%%%%%%%%%%%%%%%%%%%%%%%%%%%%%%%%%%%%%%%%%%%%%%%%%%%%%%%%%%%%
%% The "title" command has an optional parameter,
% \title{Genetic Algorithms for Timetabling: The current state of the art}
\title{Overview of Genetic Algorithms in Educational Timetabling}

%% AUTHORS %%%%%%%%%%%%%%%%%%%%%%%%%%%%%%%%%%%%%%%%%%%%%%%%%%%%%%%%%%%%%%%%%%%%%
\author{Luca Quaer}
\email{luca@quaer.net}

\affiliation{
  \institution{Wilhelm Büchner Hochschule}
  \city{Darmstadt}
  \state{Baden-Württemberg}
  \country{Germany}
}

%% DATES %%%%%%%%%%%%%%%%%%%%%%%%%%%%%%%%%%%%%%%%%%%%%%%%%%%%%%%%%%%%%%%%%%%%%%%
% \received{20 February 2007}
% \received[revised]{12 March 2009}
% \received[accepted]{5 June 2009}


%% ABSTRACT %%%%%%%%%%%%%%%%%%%%%%%%%%%%%%%%%%%%%%%%%%%%%%%%%%%%%%%%%%%%%%%%%%%%

%% The abstract is a short summary of the work to be presented in the
%% article.
\begin{abstract}
Empty
\end{abstract}


%% KEYWORDS %%%%%%%%%%%%%%%%%%%%%%%%%%%%%%%%%%%%%%%%%%%%%%%%%%%%%%%%%%%%%%%%%%%%
%% Keywords. The author(s) should pick words that accurately describe
%% the work being presented. Separate the keywords with commas.
\keywords{Genetic Algorithms, Educational Timetabling, Metaheuristics}


%% TITLE %%%%%%%%%%%%%%%%%%%%%%%%%%%%%%%%%%%%%%%%%%%%%%%%%%%%%%%%%%%%%%%%%%%%%%%
\maketitle


%% CONTENT %%%%%%%%%%%%%%%%%%%%%%%%%%%%%%%%%%%%%%%%%%%%%%%%%%%%%%%%%%%%%%%%%%%%%

%% Introduction ------------------------------------------------------------- %%
\section{Introduction}
% What is educational timetabling
Educational timetabling involves creating schedules for educational
institutions such as schools, colleges, and universities.
The problem domain can be divided into the following three main problems
\cite{kingston2013educational,schaerf1999survey}:
High-School Timetabling (HTT), University Course Timetabling (CTT) and
University Examination Timetabling (ETT).
Although a clear distinction between these three problems is not always
possible, they generally differ significantly from one another
\cite{Beligiannis2009}.
However, each of these problems essentially is a resource allocation problem
with the goal of assigning classrooms, instructors, and students to specific
time slots for various courses or activities, ensuring that all constraints and
requirements are met.
This includes avoiding conflicts (e.g., a student being scheduled for two
classes at the same time), adhering to institutional policies, and maximizing
the efficient use of resources.

% Why is it complicated?
The difficulty in finding a valid and effective solution to such a problem
lies in meeting the diverse requirements of different stakeholders
(e.g. students, teachers, administration), multiple constraints and resolving
resource conflicts in a combinatorial complex solution space caused by the
numerous constraints.
Timetabling problems like these are therefore known to be NP-complete in their
general form, meaning that the difficulty of finding a solution increases
exponentially with the problem size, which in turn makes it impossible to find
a deterministic algorithm providing an acceptable solution in polynomial time
\cite{Beligiannis2009}.
%
% What are solutions to the problem?
One popular approach to addressing the complexity of timetabling problems is the
use of metaheuristic algorithms \cite{Beligiannis2009}.
This class of algorithms leverages a non-deterministic search approach which
compromises on finding an optimal solution in favor of better runtime
performance. Consequently, such algorithms are not guaranteed to find
the best solution for a given problem, but a near optimal one
\cite{Affenzeller2009}.
Despite this limitation, metaheuristic algorithms are widely used in educational
timetabling due to their ability to provide high-quality solutions within
a reasonable timeframe.
These algorithms can be broadly classified into two categories: single-solution
and population-based metaheuristics \cite{Katoch2021}.
Single-solution based algorithms use a single candidate solution and iteratively
improve it by using local search, but are prone to get stuck in local maxima
\cite{Katoch2021}.
Population-based metaheuristics on the other hand work on multiple candidate
solutions during the search process, avoiding the risk of getting stuck
in a local maximum by maintaining diversity among the solution candidates
\cite{Katoch2021}.
Popular single-solution based algorithms in the timetabling domain are
simulated annealing, local search and Tabu search
\cite{Ceschia2023, Katoch2021}.
Well-known population based metaheuristics are genetic algorithms, particle
swarm optimization and ant colony systems \cite{Beligiannis2009, Katoch2021}.

% Solving problems of such complexity in a practicable timeframe is therefore
% mostly done by using metaheuristic algorithms, which find solutions to
% optimization problems but are not guaranteed to find the best solution.
% Therefore, optimization algorithms trying to find (near) optimal solutions have
% been used as alternative methods of solving educational timetabling problems.
% Examples for such methods are local search and simulated annealing techniques
% as well as computational intelligence algorithms like genetic algorithms (GAs),
% Tabu Search, Ant Systems and other metaheuristic approaches
% \cite{Beligiannis2009}.

% The algorithm of interest; Why are GAs well suited for timetabling?
Among these methods, genetic algorithms are known for their versatility and
application in a variety of use cases with the need of searching for solutions
of a combinatorial problem in a large solution space.
Therefore, this paper specifically focuses on genetic algorithms and how they
are used in the domain of educational timetabling.

% What is a GA?
Genetic algorithms (abbr. \textit{GA}) are a heuristic search method inspired by
the process of natural selection in biological evolution and thus belong to
the group of evolutionary algorithms \cite{Katoch2021}. As mentioned previously,
genetic algorithms utilize a population based approach, meaning multiple
solution candidates are iteratively evolved through numerous generations
imitating the Darwinian theory of survival of the fittest \cite{Katoch2021}.

% Introduce structure



%% Methods ------------------------------------------------------------------ %%
\section{Methods}
To do.

%% Basic Concepts ----------------------------------------------------------- %%
\section{Basic Concepts}
% Introduction into GAs

% Diagram of the different GAs and how they can be classified/grouped.

To do.
\subsection{A}
Sub A

\subsection{B}
Sub B


\section{Advanced Techniques}
To do.

\section{Discussion}
To do.

\section{Conclusion}
To do.

\section{Future Work}
To do.


%% ACKNOWLEDGEMENTS %%%%%%%%%%%%%%%%%%%%%%%%%%%%%%%%%%%%%%%%%%%%%%%%%%%%%%%%%%%%
% \begin{acks}
% To Robert, for the bagels and explaining CMYK and color spaces.
% \end{acks}

%% BIBLIOGRAPHY %%%%%%%%%%%%%%%%%%%%%%%%%%%%%%%%%%%%%%%%%%%%%%%%%%%%%%%%%%%%%%%%
\bibliographystyle{ACM-Reference-Format}
\bibliography{main}


%% APPENDIX %%%%%%%%%%%%%%%%%%%%%%%%%%%%%%%%%%%%%%%%%%%%%%%%%%%%%%%%%%%%%%%%%%%%
% \appendix
% \section{Research Methods}


%% END %%%%%%%%%%%%%%%%%%%%%%%%%%%%%%%%%%%%%%%%%%%%%%%%%%%%%%%%%%%%%%%%%%%%%%%%%

\end{document}
\endinput

%%%%%%%%%%%%%%%%%%%%%%%%%%%%%%%%%%%%%%%%%%%%%%%%%%%%%%%%%%%%%%%%%%%%%%%%%%%%%%%%
